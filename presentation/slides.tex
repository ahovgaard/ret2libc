\documentclass{beamer}

\usepackage[utf8]{inputenc}
\usepackage{url}

\usetheme{Copenhagen}
\usecolortheme{default}

\title{The advanced return-into-lib(c) exploits}
\subtitle{\url{http://phrack.org/issues/58/4.html}}
\author{Anders Kiel Hovgaard \and Daniel Gavin \and Rúni Klein Hansen}
\institute{Department of Computer Science, University of Copenhagen}
\date{May 22, 2015}

\begin{document}

\frame{\titlepage}

\section{Classical return-into-libc} % Anders

\begin{frame}[fragile]
  \frametitle{Classical return-into-libc}

  A method commonly used to circumvent non-executable stack by returning to a
  dynamic library instead of returning to code located on the stack.

  \begin{verbatim}
    |         ...        |  arg_2
    |--------------------|
    | addr. of "/bin/sh" |  arg_1
    |--------------------|
    |  dummy ret. addr.  |  dummy_int32
    |--------------------|
    | addr. of system()  |  funcion_in_lib
    |--------------------|
    |     0x41414141     |  buffer fill-up
    |     0x41414141     |
    |         ...        |
    |--------------------|
  \end{verbatim}
\end{frame}


\section{Chaining return-into-libc calls} % Anders

\subsection{Problems with the classical approach}

\begin{frame}

\end{frame}


\section{PaX features}  % Rúni


\frame{Demo!}



\section{The dynamic linker's dl-resolve() function}  % Daniel





\end{document}
