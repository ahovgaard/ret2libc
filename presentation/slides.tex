\documentclass{beamer}

\usepackage[utf8]{inputenc}
\usepackage{url}

\usetheme{Copenhagen}
\usecolortheme{default}

\title{The advanced return-into-lib(c) exploits}
\subtitle{\url{http://phrack.org/issues/58/4.html}}
\author{Anders Kiel Hovgaard \and Daniel Gavin \and Rúni Klein Hansen}
\institute{Department of Computer Science, University of Copenhagen}
\date{May 22, 2015}

\begin{document}

\frame{\titlepage}

\section{Classical return-into-libc}

\begin{frame}[fragile]
  \frametitle{Classical return-into-libc}

  A method commonly used to circumvent non-executable stack by returning to a
  dynamic library instead of returning to code located on the stack.

  \begin{verbatim}
    |         ...        |  arg_2
    |--------------------|
    | addr. of "/bin/sh" |  arg_1
    |--------------------|
    |  dummy ret. addr.  |  dummy_int32
    |--------------------|
    | addr. of system()  |  funcion_in_lib
    |--------------------|
    |     0x41414141     |  buffer fill-up
    |     0x41414141     |
    |         ...        |
    |--------------------|
  \end{verbatim}
\end{frame}


\section{Chaining return-into-libc calls} % Anders

\subsection{Problems with the classical approach}

\begin{frame}
  \frametitle{Problems with the classical approach}
  \begin{itemize}
    \item Can't generally call another argument taking function after the firt
      call, since the first argument will be the new return address etc.
      Often necessary, e.g. for:
      \begin{itemize}
        \item regaining privileges with \texttt{setuid}
        \item mapping a known memory location with \texttt{mmap}
      \end{itemize}

    \item The overflow can typically not contain \texttt{NUL} bytes and that
      limits the arguments to the function.
      \begin{itemize}
        \item \texttt{mmap(0x414140\textcolor{red}{00}, \dots)} \hfill
          (pagesize = 0x1000)
      \end{itemize}
  \end{itemize}
\end{frame}

\begin{frame}
  \frametitle{Chaining return-into-libc calls}
  Two methods for chaining multiple function calls:
  \begin{itemize}
    \item ``\texttt{esp} lifting'' method
    \item frame faking
  \end{itemize}
\end{frame}

\subsection{``\texttt{esp} lifting'' method}

\begin{frame}
  \frametitle{``\texttt{esp} lifting'' method}

\end{frame}


\section{PaX features}  % Rúni


\frame{Demo!}



\section{The dynamic linker's dl-resolve() function}  % Daniel





\end{document}
