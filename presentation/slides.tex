\documentclass[10pt]{beamer}

\usepackage[utf8]{inputenc}
\usepackage{url}
\usepackage{listings}
\usepackage{drawstack}

\lstset{
  basicstyle=\ttfamily\scriptsize,
  showtabs=false,
  showspaces=false,
  showstringspaces=false,
  columns=fixed,
  showstringspaces=false,
  extendedchars=true,
}

\usetheme{Copenhagen}
\usecolortheme{default}

\title{The advanced return-into-lib(c) exploits}
\subtitle{\url{http://phrack.org/issues/58/4.html}}
\author{Anders Kiel Hovgaard \and Daniel Gavin \and Rúni Klein Hansen}
\institute{Department of Computer Science, University of Copenhagen}
\date{May 22, 2015}

\begin{document}

\frame{\titlepage}

\section{Classical return-into-libc}  % Rúni

\begin{frame}[fragile]
  \frametitle{Classical return-into-libc}

  A method commonly used to circumvent non-executable stack by returning to a
  dynamic library instead of returning to code located on the stack.

  \begin{verbatim}
    |         ...        |  arg_2
    |--------------------|
    | addr. of "/bin/sh" |  arg_1
    |--------------------|
    |  dummy ret. addr.  |  dummy_int32
    |--------------------|
    | addr. of system()  |  funcion_in_lib
    |--------------------|
    |     0x41414141     |  buffer fill-up
    |     0x41414141     |
    |         ...        |
    |--------------------|
  \end{verbatim}
\end{frame}


\section{Chaining return-into-libc calls} % Anders

\subsection{Problems with the classical approach}

\begin{frame}[fragile]
  \frametitle{Problems with the classical approach}
  \begin{itemize}
    \item Not possible to call another funtion, which takes arguments, after
      the first call, since the first argument will be the new return address
      etc.
  \end{itemize}

  \begin{lstlisting}
  -------------------------------------------------------
  | buffer fill-up | f1 | f2 | arg_1/f2_ret | arg_2 | ...
  -------------------------------------------------------
  \end{lstlisting}

  \begin{itemize}
    \item Multiple function calls often necessary, e.g. for:
      \begin{itemize}
        \item regaining privileges with \texttt{setuid}
        \item mapping a known memory location with \texttt{mmap}
        \item copying or reading code to mapped location
        \item returning to mapped location
        \item etc.
      \end{itemize}
  \end{itemize}

\end{frame}

\begin{frame}
  \frametitle{Problems with the classical approach}
  \begin{itemize}
    \item The overflow can typically not contain \texttt{NUL} bytes and that
      limits the arguments to the function.\\
      \hfill\\
      Example:
      \begin{itemize}
        \item \texttt{mmap(0x414140\textcolor{red}{00},
          0x20\textcolor{red}{00}, \dots)}
          \hfill (pagesize = 0x1000)
      \end{itemize}
  \end{itemize}
\end{frame}

\begin{frame}
  \frametitle{Chaining return-into-libc calls}
  Two methods for chaining multiple function calls:
  \begin{itemize}
    \item ``\texttt{esp} lifting'' method
    \item frame faking
  \end{itemize}
\end{frame}

\subsection{``\texttt{esp} lifting'' method}

\begin{frame}[fragile,shrink]
  \frametitle{``\texttt{esp} lifting'' method}

  \begin{columns}[c]
  \column{0.5\textwidth}

  \begin{itemize}
    \item Designed for attacking binaries compiled with the
      \texttt{-fomit-frame-pointer} flag.

    \item Using gadgets that ``lift'' \texttt{ESP} to clean up arguments on the
      stack in between function calls.
  \end{itemize}

  \begin{lstlisting}
  eplg:
      add esp, SIZE
      ret
  \end{lstlisting}

  \begin{lstlisting}
  eplg:
      pop ebx
      pop esi
      pop edi
      pop ebp
      ret
  \end{lstlisting}

  \column{0.5\textwidth}

  \begin{drawstack}[scale=0.54]
    \cell{f2\_args}
    \cell{dummy}
    \cell{f2}
    \startframe
    \cell{(padding)}
    \cell{f1\_argn}
    \cell{\dots}
    \cell{f1\_arg2}
    \cell{f1\_arg1}
    \finishframe{{\scriptsize SIZE}\phantom{XXX}}
    \cell{epilogue}
    \cell{f1}
    \cell{0x41414141}
  \end{drawstack}

  %\begin{lstlisting}
  %|        ...         |
  %|--------------------|
  %|       f2_args      |
  %|--------------------|
  %|  dummy ret. addr.  |
  %|--------------------|
  %|         f2         |
  %|--------------------| <------\
  %|      (padding)     |        |
  %|--------------------|        |
  %|       f1_argn      |        |
  %|--------------------|
  %|         ...        |    VARS_SIZE
  %|--------------------|
  %|       f1_arg2      |        |
  %|--------------------|        |
  %|       f1_arg1      |        |
  %|--------------------| <------/
  %|      epilogue      |
  %|--------------------|
  %|         f1         |
  %|--------------------|
  %|     0x41414141     |  buffer fill-up
  %|         ...        |
  %\end{lstlisting}
  \end{columns}
\end{frame}

\subsection{frame faking}

\begin{frame}[fragile]
  \frametitle{frame faking}
  Designed to attack binaries compiled \emph{without} the
  \texttt{-fomit-frame-pointer} flag.

  \begin{lstlisting}
  leaveret:
        leave
        ret
  \end{lstlisting}

  \texttt{ESP} lifting epilogues might still be available with GCC.

\end{frame}



\section{PaX features}  % Rúni


\frame{Demo!}



\section{The dynamic linker's dl-resolve() function}  % Daniel





\end{document}
